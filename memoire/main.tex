\documentclass{article}
\usepackage[english]{babel}
\usepackage[T1]{fontenc}
\usepackage{times}
\usepackage[utf8]{inputenc}
\usepackage[titletoc,toc,title]{appendix}

\title{Data-flow programming}
\author{Gautier DI FOLCO}
\date{2014}

\begin{document}
\maketitle

\begin{abstract}
    Data-flow programming is a familly of data-centered design patterns
    ...
\end{abstract}

\section{Introduction}\label{introduction}


\section{Flow-based}\label{flow-based}

\subsection{Main features}\label{principes}


\subsection{History}\label{historique}


\subsection{Current implementations}\label{implementations}


\subsection{Benefits and Main concerns}\label{benefices}


\subsection{Drawbacks}\label{desavanatages}


\subsection{Current researches}\label{recherches}


\section{Reactive}\label{reactive}

\subsection{Main features}\label{principes-1}


\subsection{History}\label{historique-1}


\subsection{Current implementations}\label{implementations-1}


\subsection{Benefits and Main concerns}\label{benefices-1}


\subsection{Drawbacks}\label{desavanatages-1}


\subsection{Current researches}\label{recherches-1}


\section{Conclusions}\label{conclusions}


\begin{appendices}
\section{References}\label{references}
\bibliographystyle{plain}
\bibliography{refs}
\end{appendices}
\end{document}
