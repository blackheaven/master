\documentclass{article}
\usepackage[frenchb]{babel}
\usepackage[T1]{fontenc}
\usepackage{times}
\usepackage[utf8]{inputenc}
\usepackage[titletoc,toc,title]{appendix}

\title{Caractérisation du \emph{flow-based programming}}
\author{Gautier DI FOLCO}
\date{2014}

\begin{document}
\maketitle
\tableofcontents

% \section{Abstract}\label{abstract}
\begin{abstract}
L'objectif de ce document est de caractériser le \emph{flow-based programming}.

Il va s'agir de poser un lexique réutilisé tout au long de l'étude ainsi que les
éléments primaires qui constituent le \emph{flow-based programming} et les problèmes
qu'ils impliquent.
\end{abstract}

\section{Lexique}
\subsection{\emph{flow-based programming}}
Structuration d'un programme en partant du principe que les données sont modélisées
dans un flux et qu'elles sont traités par un ensemble de noeuds.

\subsection{Flux}
Ensemble de messages ordonnés reçus par un noeud.

Un flux possède un début inaccessible et aucune fin, Il est dôté de deux attributs :
le type des messages et le débit.

\subsection{Message}
Un message possède un type et un corps.

\subsection{Noeud}
Unité de traitement qui possède des entrées qui sont des flux, où sont déposés
les messages et qui peut envoyer d'autres messages.

\subsubsection{Noeud initial}
Noeud qui ne reçoit pas de messages.

\subsubsection{Noeud terminal}
Noeud qui n'envoie pas de messages.

\subsection{Envoi de messages}
Lorsqu'un noeud envoie un message, le message est déposé par l'emetteur "chez" le
récepteur (\emph{push}).
L'emetteur reste bloqué jusqu'à ce que le destinataire ai traité ce message (c'est-à-dire
que le noeud se termine où qu'il tente de lire un autre message sur ce flux).

Lorsque l'émetteur est suceptible d'envoyer des messages à un récepteur, ont dit
qu'ils sont reliés par unie arête partant de l'émetteur au récepteur.

\subsection{Arité de noeud}
Nombre de flux attendu par un noeud.

Un noeud d'arité 1 attendra un flux, un noeud d'arité 2 attendra
deux flux, etc.

Lorsqu'un noeud d'arité supérieure à 1 ne possède plus de messages à consommer sur
tous ses flux, il va commencer à effectuer sa tâche et ne surveillera plus que les
flux qui lui manque pour la compléter.

\section{Noeuds primaires}
\subsection{Famille sans état}
\subsubsection{Identity}
Le noeud ne fait que transmettre le message sans le modifier.

\subsubsection{Map}
Le noeud change le type et le contenu du message.

\subsubsection{Tee}
Le noeud transfert le message à deux autres noeuds sans le modifier.

\subsubsection{Filter}
Le noeud transfert ou non le message en fonction d'une condition.

\subsubsection{Buffer}
Le noeud met les messages dans une file d'attente et les transferts au fur et à
mesure que le noeud suivant est en mesure de les consommer.

\subsection{Famille à état}
Il s'agit d'une famille de noeuds ont besoin de connaitre le précédent message qu'ils
ont envoyés pour calculer le message courant à envoyer.

Ils envoient les évolutions de leur calcul au fur et à mesure des messages reçus

\subsubsection{Fold}
Le noeud va calculer une information à partir d'un ensemble de messages.

\subsubsection{Best}
Le noeud va constament renvoyer le meilleur message vis-à-vis d'une condition.

\section{Cas pratique : monitoring d'un système basé sur le modèle à acteurs}
Nous avons un système à acteurs, chaque acteur est un processus indépendant qui
reçoit des messages dans sa boîte aux lettres et peut envoyer des messages à d'autres
acteurs.

Il s'agit d'effectuer du monitoring sur ce système pour être en mesure d'améliorer
ses performances.

Pour cela nous prennons un ensemble de métriques :
\begin{itemize}
    \item Messages envoyés
    \item Nombre de messages en attente
    \item Mémoire consommée par acteur
    \item Créations d'acteurs
    \item Morts d'acteurs
\end{itemize}

Certaines de ces données sont envoyées de manière périodique (comme la consommation
mémoire ou le nombre de messages en attente) alors que les autres sont envoyées au
fur et à mesure.

L'objectif est de visualiser graphiquement (sous plusieurs formes et points de vue)
le système de manière dynamique.

\end{document}
