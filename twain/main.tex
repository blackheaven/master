\documentclass{article}
\usepackage[frenchb]{babel}
\usepackage[T1]{fontenc}
\usepackage{times}
\usepackage[utf8]{inputenc}
\usepackage[titletoc,toc,title]{appendix}

\title{TWAIN\footnote{Technology Without an Interesting Name} \emph{flow-based programming}}
\author{Gautier DI FOLCO}
\date{2014}

\begin{document}
\maketitle
\tableofcontents

% \section{Abstract}\label{abstract}
\begin{abstract}
L'objectif de ce document est de fournir des pistes sur l'intégration du
\emph{flow-based programming} dans les langages impératifs.
\end{abstract}

\section{Constats}
\subsection{\emph{Legacy}}
Aujourd'hui la majorité des programmes existants se reposent sur le paradigme impératif.

Cela implique que d'une part la majorité des développeurs est habituée au paradigme
impératif.
Changer de paradigme est un effort intellectuel que tout le monde ne peut pas fournir,
le temps d'apprentissage global (c'est-à-dire de la syntaxe, des concepts et des
idiomes) et non-négligeable.
À tel point qu'il peut sembler plus rentable de vivre avec les difficultés
d'expressions intrinsèques au paradigme vis-à-vis du problème que l'on tente de
résoudre, que d'apprendre et d'utiliser un paradigme plus adéquat.

Et d'autre part la masse de code pré-existante (ainsi que les outils disponibles)
est trop importante pour envisager une ré-écriture ou une adaptation
vers ou pour des langages de type \emph{flow-based programming}.

\subsection{Objectifs du \emph{flow-based programming}}
L'un des objectif du \emph{flow-based programming} est de modéliser la notion de
continuité d'arriver de l'information.

Nous passons d'une vision de l'information complète et finie à une vision partielle
et infine.

Nous devons insérer cette notion de "non-terminaison" dans le langage afin de
pouvoir modéliser des flux continus d'information.

\section{Choix}
Nous choisissons d'ajouter une une notion d'infini, tout en provoquant le moins de
modifications possible dans les programmes existants, afin que l'intégration à
des programmes existants soit aussi transparente que possible.

\section{Paresse}
Le problème majeur c'est que dans les langages impératifs l'exécution des instructions
se fait au fur et à mesure.
Créer une liste infinie bloquerait l'ensemble du programme sur la construction de
cette liste, sans que le programme puisse d'arrêter.
Pour que celà soit possible, nous devons introduire la notion de paresse.

Ainsi, les éléments d'une liste ne seront calculés que si une instruction tente
d'accéder à un de ses éléments (ou si une boucle tente d'itérer sur cette dernière
via un \emph{iterator}).

De plus, lorsqu'une boucle va itérer sur un flux, celle-ci sera bloqué jusqu'à ce
qu'un élément soit disponible pour être traité.


\end{document}
