% LLNCStmpl.tex
% Template file to use for LLNCS papers prepared in LaTeX
%websites for more information: http://www.springer.com
%http://www.springer.com/lncs

\documentclass{llncs}
%Use this line instead if you want to use running heads (i.e. headers on each page):
%\documentclass[runningheads]{llncs}


\begin{document}
\title{\emph{Data-Flow} : Devenez le Super-Arrow du rythme}

%If you're using runningheads you can add an abreviated title for the running head on odd pages using the following
%\titlerunning{abreviated title goes here}
%and an alternative title for the table of contents:
%\toctitle{table of contents title}

\subtitle{Modélisation du rythme de traitement et du changement de structure des flux dans un contexte \emph{Data-Flow}}

%For a single author
%\author{Author Name}

%For multiple authors:
\author{Gautier DI FOLCO\inst{1}}


%If using runnningheads you can abbreviate the author name on even pages:
%\authorrunning{abbreviated author name}
%and you can change the author name in the table of contents
%\tocauthor{enhanced author name}

%For a single institute
%\institute{Institute Name \email{email address}}

% If authors are from different institutes 
\institute{INRIA \email{gdifolco@ens-lyon.fr}}

%to remove your email just remove '\email{email address}'
% you can also remove the thanks footnote by removing '\thanks{Thank you to...}'


\maketitle

\begin{abstract}
Le Data-Flow avait pour objectif initial de modéliser le continu dans les
systèmes informatiques.

Au fil des années, il a été utlisé pour faire des animations de manière à ne se
concentrer que sur les formules mathématiques des formes géométriques affichées.

Récemment, suite à l'exploision du volume de données à traiter, le \emph{Data-Flow}
est devenu un moyen simple de raisonner sur des données incomplète, c'est-à-dire,
qu'on ne peut avoir en intégralité, dont on ne connait ni le début ni la fin.

Notre objectif est de proposer des outils pour manipuler des flux ayant des
rythmes et des structures différent explicitement modélisés.
\end{abstract}

\section{Introduction}

\section{Historique}
\subsection{Réseaux de Khan}
\subsection{Prémisses de la modélisation du continu avec le \emph{Flow-based programming}}
Lucid, Lustre, Signal et Esterel
\subsection{L'avènement des environnements interactifs avec le \emph{Reactive-based programming}}
\subsection{Prise d'envol des besoins de traitements et perte de la vision complète des données}

\section{Modélisation du rythme et de la structure des flux}
\subsection{Cas pratique : box office}
\subsubsection{Modélisation sans notion de rythme}
\subsubsection{Modélisation avec de rythme}
\subsection{Mise en oeuvre avec Haskell}
Introduction à Haskell et des arrows/comonad
\subsection{Introduction des Super-arrows}

\section{Conclusion}

%The bibliography, done here without a bib file
%This is the old BibTeX style for use with llncs.cls
\bibliographystyle{splncs}

%Alternative bibliography styles:
%the following does the same as above except with alphabetic sorting
%\bibliographystyle{splncs_srt}
%the following is the current LNCS BibTex with alphabetic sorting
%\bibliographystyle{splncs03}
%If you want to use a different BibTex style include [oribibl] in the document class line

\begin{thebibliography}{1}
%add each reference in here like this:
\bibitem[RE1]{reference1}
Author:
Article/Book:
Other info: (date) page numbers.
\end{thebibliography}

\end{document}

