\documentclass{article}
\usepackage[frenchb]{babel}
\usepackage[T1]{fontenc}
\usepackage{times}
\usepackage[utf8]{inputenc}
\usepackage[titletoc,toc,title]{appendix}
\usepackage{listings}
\lstset{breaklines=true, breakatwhitespace=true}
\usepackage{varwidth}

\title{Caractérisation de la programmation fonctionnelle}
\author{Gautier DI FOLCO}
\date{2014}

\begin{document}
\maketitle
\tableofcontents

% \section{Abstract}\label{abstract}
\begin{abstract}
L'objectif est d'être en mesure de pouvoir distinguer un langage fonctionnel
d'un langage impératif.
\end{abstract}

Les exemples de codes fonctionnel sont en Haskell et les exemples de codes impératifs sont en Ruby.

\section{Caractérisations}
\subsection{Exécution}
Conceptuellement, dans un langage de fonctionnel l’exécution d’un programme se
fait uniquement par substitution d’expression sans modification de l’état d’une
machine alors que dans un langage impératif l’exécution se fait par évaluation
d’expression et modification de l’état de la machine.

Ainsi toutes les fonctions des langages fonctionnels ne sont que des expressions
qui possède ou non un nom, on parle de \emph{bind}, qui peut être substitué par
l'expression elle-même.

Une fonction non-nommée est appelée \emph{lambda expression}.

Schématiquement : toute fonction ne ferait que renvoyer une valeur.

Exemple de somme des éléments d'un tableau :

\begin{itemize}
    \item Haskell

\begin{lstlisting}[language=haskell]
foldl (+) 0 [1, 2, 3]
-- Sera execute de cette maniere
-- foldl (+) (0 + 1) [2, 3]
-- foldl (+) (0 + 1) [2, 3]
-- foldl (+) (0 + 1 + 2) [3]
-- foldl (+) (0 + 1 + 2 + 3) []
-- (0 + 1 + 2 + 3)
-- (1 + 2 + 3)
-- (3 + 3)
-- 6
\end{lstlisting}

    \item Ruby

\begin{lstlisting}[language=ruby]
sum = 0
data = [1, 2, 3]
tmp = nil
while (tmp = data.shift)
    sum = sum + tmp
end
# Sera execute de cette maniere
# sum = 0 / tmp = 1 / data = [2, 3]
# sum = 1 / tmp = 2 / data = [3]
# sum = 3 / tmp = 3 / data = []
# sum = 6 / tmp = nil / data = []
\end{lstlisting}
\end{itemize}

\subsection{Fonction comme citoyen de première classe}
Un langage semble fonctionnel dès qu'il inclut des mécanismes pour manipuler des
fonctions comme citoyens de première classe (valeurs fonctionelles, création de
fonctions anonymes, fonctions d'ordre supérieur).

\subsection{Un état dans la pile mais pas dans le tas}
L'état d'un système basé sur la programmation fonctionnelle se trouve uniquement
dans la pile, donc ni dans le tas, ni dans une variable globale.

\subsection{Pas de notions de mutabilité}
Tout n'étant qu'expression, aucune expression ne peut être modifiée.

Aucun effet de bord n'est possible.

\subsection{Transparence référentielle}
Deux appels à une même fonction avec les mêmes arguments donnera le même résultat.

\subsection{Ré-écriture des expressions}
L'une des conséquences de la transparence référentielle est la liberté que le compilateur
a de ré-écrire les expressions.

Les fonctions n'ayant pas d'effets de bords, il peut changer l'ordre d'appel s'il
considère que ça peut apporter un gain quelconque.

\subsection{Évaluation paresseuse}
Le second bénéfice de la transparence référentielle est la possibilité de n'exécuter
que les expressions qui sont nécessaires.

Ainsi cela permet d'introduire la notion d'infini.

Par exemple, nous pouvons définir la suite de fibonacci sans se préoccuper de sa
terminaison :

\begin{lstlisting}[language=haskell]
fib = 1:1:zipWith (+) fib (tail)
-- n'evaluera rien tant qu'il n'y est pas force
fib !! 3
-- renverra 5
-- seul les 4 premiers elements sont evalues
\end{lstlisting}

\subsection{Pureté/Impureté}
Afin de rendre nos programme utile, il faut être en mesure d'effectuer
des effets de bords afin de pouvoir, par exemple, lire des informations en entrée
et écrire le résultat des calculs en sortie.

On considère qu'un langage est pure lorsque ces effets de bords sont effectués au
travers d'une structure comme une monade\footnote{un ensemble de types
encapsulants une valeur dans un contexte, fournissants des fonctions pour
modifier cette valeur, cette modification étant retranscrite de manière
impérative vers la cible, modifiant le contexte par la même occasion}.

Les langages ne faisant pas cette distinction sont considérés comme impurs ou strict.

\subsection{Paramètres et currification}
Les fonctions ne peuvent prendre et retourner qu'une valeur.

Ainsi, pour qu'une fonction puisse prendre plusieurs valeurs, chaque fois qu'une
valeur sera fournie en paramètre, elle va conceptuellement créer une \emph{lambda expression},
potentiellement appelée \emph{closure} si son corps dépend d'un des paramètres
passé lors de sa création qui va attendre le paramètre suivante de la fonction à
plusieurs paramètres, etc.
Jusqu'à ce qu'aucun paramètre ne manque.

\begin{lstlisting}[language=haskell]
add3 :: Int -> Int -> Int -> Int
add3 a b c = a + b + c
-- ou add3 = (\a -> (\b -> (\c -> (a + b + c))))

add3 -- Int -> Int -> Int -> Int
add3 1 -- Int -> Int -> Int
add3 1 2 -- Int -> Int
add3 1 2 3 -- Int
\end{lstlisting}

On parle de currification ou \emph{currying}.

Ainsi le fait ne fournir qu'une partie des paramètres attendus créera une fonction
valide et manipulable par le programme.

\end{document}
