\documentclass{article}
\usepackage[frenchb]{babel}
\usepackage[T1]{fontenc}
\usepackage{times}
\usepackage[utf8]{inputenc}
\usepackage[titletoc,toc,title]{appendix}

\title{\emph{Data-flow programming}}
\author{Gautier DI FOLCO}
\date{2014}

\begin{document}
\maketitle
\tableofcontents

% \section{Abstract}\label{abstract}
\begin{abstract}
Le \emph{data-flow programming} est une approche de structuration des programmes
ayant pour but de considerer que les données dont il dispose sont sous forme
d'un ou plusieurs flux d'informations continus.
Il s'agit donc de ne plus centrer le programme sur les traitements qu'il va
effectuer mais sur la circulation des données qu'il manipule.

Le but de ce document est d'introduire les concepts du \emph{data-flow programming}
sous la forme d'une synthèse bibliographique, en proposant quelques critères de
comparaison des grandes familles de langage \emph{data-flow}.

Il s'agit également de voir quels changements structurels le \emph{data-flow}
entraine dans les systèmes qui se basent sur cette approche.
\end{abstract}

% TODO : reformatter
% objectif : chercher à savoir quels sont les changements de structures apportés par le DF
% gain/pertes

% contributions : à définir

    % exemple de monitoring : le flux était là avant, mais pas monitoré
    % comment on le ferait en impératif, en fp et en DF ?
    % définir les trois avant d'expliquer comment le faire

    % détail du concept général
    % bien décrire un flux + flux tiré/poussé
    % description DÉTAILLÉ (autre doc) du fb (+ noeuds) du FB
    % idem rp
    % lien entre rp et fb (ci dessous)

% rp => vient du fp
% fb => vient de l'impératif
    % fb : 0 tools
    % rp set on tool to pre-proc flow
    % fb : has to known its output
    % rp : emet un event et le système de débrouille
\section{Introduction}\label{introduction}

\subsection{Cas pratique}
Nous souhatons mettre en place un ensemble de mécanismes de surveillances sur un
système quelconque.

Dans cette optique, nous allons commencer par ajouter un certain nombre de
sondes pour nous faire remontrer les métriques que nous voulons surveiller.

Puis nous allons appliquer un ensemble de modifications et d'aggrégation des
données récoltées pour obtenir des données qui correspondent à notre objectif
de modélisation.

Enfin nous allons produire des rapports de ce système à un instant donnée mettant
en avant les caractéristiques de celui-ci.

\subsection{Approches de résolution}
\subsubsection{Impérative}
La base de la programmation impérative est la spécification séquentielle
des calculs à effectuer sur une donnée d'entrée pour obtenir une donnée en sortie.

Nous allons disposer d'un ensemble de données à un endroit de notre mémoire et
nous lui appliquons une suite d'opération destructives pour lui donner une forme
plus approprié à notre usage général de celle-ci dans tout le programme.

Ainsi, pour atteindre notre objectif, nous allons avoir une première partie de
récupération des données qui va se charger d'obtenir les métriques du système
surveillé à un instant donné et faire un traitement minimal pour les structurer
de manière à ce qu'elles soient manipulable simplement dans tout le reste du programme.

Dans un second temps les données vont subir un ensemble de modifications, de
taitements et d'aggregations, notament avec les traitements d'anciennes métriques,
pour obtenir un ensemble de données modélisée de sorte à ce qu'elles mettent en
valeur l'état du système que l'utilisateur veut observer.

Enfin cet état dans le temps du système sera communiqué à l'utilisateur sous forme
de diagrammes par exemple.

L'utilisateur aura donc une vue du passée du système surveillé jusqu'à un moment donné.

\subsubsection{Fonctionnelle}
Lorsque l'on architecture un programme basé sur la programmation fonctionnelle,
on spécification de l'ensemble des transformations à effectuer sur des données
d'entrées pour obtenir les données de sortie.

Conceptuellement, la donnée ne change pas, un certain nombre de filtres et de
changes à opérer pour obtenir une nouvelle donnée sous un autre format.

Pour traiter notre cas pratique, nous programme va recevoir les métriques et
les stockers dans un format proche de leur formattage brute, aucun traitement
ne sera effectué.

Ensuite, ces métriques, se verront appliquer un certain nombre de filtres pour
otenir des données formattées pour l'aggrégation.
Puis, elles seront agrégées avec les anciennes métriques formattées pour l'aggrégation.
Elles seront ensuite filtrées et transformées por obtenir des données propres à
la communication avec l'utilisateur.

Enfin les données seront affichées à l'utilisateur qui aura une vue du passée du
système surveillé jusqu'à un moment donné.

\subsubsection{Flux}
L'approche flux vise à spécifier l'enchainement des transformations à partir
d'un flux de données d'entrée pour obtenir une flux de données en sortie.

Un flux est un ensemble de messages dont le début est inaccessible, ce qui
implique que nous n'aurons jamais l'ensemble des messages.
Un flux n'a pas de fin, ce qui signifie que nous ne pouvons pas obtenir une
image fixe du monde décrit par le flux ayant une valeur constante dans le temps.
Plus l'information est ancienne, moins elle a de valeur puisque le monde avance
en permanence.

Pour résoudre notre problème, nous allons faire un ensemble de parties de programme
chargées de récupérer les flux de métriques entrants et de les transmettres,
sous forme de flux modifiés, à d'autres parties du programme chargées de combiner,
filtrer ou aggréger ces métriques.
Ce qui entrainera la création de nouveaux flux qui seront affichés en temps-réel
à l'utilisateur.

Ainsi, l'utilisateur aura une vu du système présent, mais aucun historique.

\subsection{Caractérisation}
L'approche flux décrite précédemment se nomme \emph{data-flow programming}~\cite{dataflow}.

Le \emph{data-flow} est une famille de \emph{design patterns} structurels,
il s'agit d'articuler le programme autour de flux infini de données.
Ces flux sont manipulés par des noeuds qui vont effectuer des traitements sur
les données de ces flux, ce qui va générer un nouveau flux de données.

Le programme est réduit à un ensemble de noeuds qui
manipulent les données et des arcs qui représentent les
communications entre ces noeuds.

Il existe deux sous-familles de langages de programmation \emph{data-flow} :
\emph{flow} et \emph{reactive}.

Le \emph{flow-based programming} consiste à recevoir, traiter et envoyer des messages de
manière directe entre les parties d'un système.

Le \emph{reactive-based programming} consiste à récupérer, appliquer des comportements
et émettre et des événements dans le systèmes.

Le \emph{flow-based programming} et le \emph{reactive-based programming} ont comme
caractéristique commune de réagir dans le temps à une entrée qui n'a pas de fin et
dont le début est inaccessible.

Afin de comparer ces deux familles, arrêtons-nous d'abord sur quelques critères.
Ces critères nous permettront de cibler le comportement des différentes familles présentes.

\section{Critères}\label{criteres}
\subsection{Composabilité}
La composabilité est la faciliter de modification de parties du système qui
communiquent entre elles.

Plus le nombre de changements est important pour faire communiquer deux parties
d'un système qui ne communicaient pas entre elles, plus la composabilité est mauvaise.
Inversement, moins le nombre de changements est important, meilleure est la composabilité.

\subsection{Évolutivité}
L'évolutivité est la facilité d'ajout de nouvelles parties dans un système et notamment
la capacité à s'intégrer facilement avec les parties déjà en place.

Plus l'ajout de nouvelles parties au système communcant avec les anciennes parties
de ce même système nécessitera peu de changement, meilleure sera l'évolutivité.
À l'opposé, plus le nombre de modifications nécessaires est important, plus l'évolutivité
est mauvaise.

\subsection{Reproductibilité}
La reproductibilité est la capacité du système à produire le même ensemble de messages,
dans le même ordre et avec les mêmes informations, à partir des mêmes informations d'entrées,
ce qui doit donner les mêmes résultats en sortie.


\section{\emph{Flow-based programming}}\label{flow-based}

\subsection{Principes}\label{principes}

Le \emph{flow-based programming} reprend cette idée de noeuds
communicants. Les communications se font par passage de message, de noeud à noeud,
sans intermédiaire.

\subsection{Historique}\label{historique}
L'exemple le plus important de \emph{flow-based programming} est le modèle à acteurs~\cite{actors}
qui a été largement diffusé par Erlang~\cite{erlang}.
Erlang a comme principal objectif de fournir une infrastructure hautement disponible
pour le monde des télécommunications.
Pour y parvenir il a été nécessaire de changer la manière dont les programmes sont pensés.
Il ne s'agit plus de tenter de réduire les erreurs, il faut au contraire les gérer.
Il faut être en mesure de faire des reprises après erreurs ou après un panne.
Ainsi il faut fournir à un langage les primitives nécessaires pour, non seulement
être en mesure de faire communiquer les différentes partie du système, mais aussi
pour garantir les bonnes communications et le bon fonctionnement global du système.

Pour cela le modèle à acteurs a été créé. Le principe de base est d'avoir un grand
nombre de tâches s'exécutant de manière concurrente et non déterministes.
Une tâche est appelée acteur et est l'unique propriétaire de ses données.
Les acteurs communiquent entre eux par envois de messages.
Lorsqu'un message est reçu par un acteur il est ajouté à sa file d'attente~\footnote{il s'agit du principe d'une boîte aux lettres}.
Les messages sont donc traités un à un.

Lorsqu'un acteur est indisponible, il est possible de continuer à lui envoyer
des messages.
Un acteur peut être indisponible pour les raisons suivantes :
\begin{itemize}
    \item Il s'est arrêter, normalement ou non
    \item Son code est en train d'être rechargé à partir d'une version différente
    \item Il n'est pas en cours d'exécution
\end{itemize}

Cela permet d'être en mesure de changer les acteurs de coeur dans le processeur ou
de machine, en chargeant une nouvelle version du code de l'acteur ou de le relancer
après une erreur, sans perdre d'informations et sans devoir suspendre le système.

Plus tard des implantations en bibliothèques ont vu le jour comme Akka~\cite{akka}
en Scala et celluloid~\cite{celluloid} en Ruby.

\subsection{Noeuds primaires}
Les seules capacités du noeud sont la réception et l'émission pair-à-pair de messages.

Il existe cependant un nombre de motifs récurrents décrits ci-dessous.

\subsubsection{Famille sans état}
\paragraph{\emph{Identity}}
Le noeud ne fait que transmettre le message sans le modifier.

\paragraph{\emph{Map}}
Le noeud change le type et le contenu du message.

\paragraph{\emph{Tee}}
Le noeud transfert le message à deux autres noeuds sans le modifier.

\paragraph{\emph{Filter}}
Le noeud transfert ou non le message en fonction d'une condition.

\subsubsection{Famille à état}
Il s'agit d'une famille de noeuds ont besoin de connaitre le précédent message qu'ils
ont envoyés pour calculer le message courant à envoyer.

Ils envoient les évolutions de leur calcul au fur et à mesure des messages reçus

\paragraph{\emph{Fold}}
Le noeud va calculer une information à partir d'un ensemble de messages.

\paragraph{\emph{Best}}
Le noeud va constament renvoyer le meilleur message vis-à-vis d'une condition.

\paragraph{\emph{Buffer}}
Le noeud met les messages dans une file d'attente et les transferts au fur et à
mesure que le noeud suivant est en mesure de les consommer.

\paragraph{\emph{Join}}
Le noeud attends la réception de plusieurs messages qu'il va concaténer en un seul,
puis transférer le message créé à son successeur.

\subsection{Bénéfices}\label{bénéfices}
Le système est relativement simple à mettre en oeuvre puisqu'il se content de mettre
en relation de manière directe et explicite des noeuds qui se connaissent.

\subsection{Comparaisons aux critères}
\begin{itemize}
    \item[Composabilité] Aucune aide n'est apportée à la composabilité, cela dépendra
de la capacité de ceux qui façonnent le système à respecter des conventions et à
faire des interface suffisament abstraite. Ce qui implique une composabilité variable
généralement mauvaise.
    \item[Évolutivité] Lorsqu'une nouvelle partie doit être ajoutée à un système
utilisant cette approche, il faut identifier tous les points de liaisons et faire
en sorte de s'assurer que notre ajout ne viennent pas perturber leur bon fonctionnement,
ce qui est coûteux et donne une mauvaise évolutivité.
    \item[Reproductibilité] Il n'y a aucune reproductibilité du fait de l'apect
non-déterministe de la réception des messages par les différentes parties du système.
\end{itemize}


\section{\emph{Reactive-based programming}}\label{reactive}

\subsection{Principes}\label{principes-1}

La programmation vise à réagir à des événements en appliquant des
comportements sur ceux-ci. Ces événements sont émits dans le système et celui-ci
fait en sorte de faire parvenir ces événéments à toutes les parties qui ont un comportement
à lui appliquer.

\subsection{Historique}\label{historique-1}

Haskell~\cite{haskell} est un langage de programmation fonctionnelle pure, c'est-à-dire
que chaque partie d'un programme est considérée comme une expression mathématique
à évaluer. L'évaluation se fait par substitution~\footnote{comme le fait le pré-processeur
des compilateurs du langage C}, il y n'a donc aucune mutation ou changement d'état
possible. Cela conduit à une propriété appelée transparence référentielle, qui se
manifeste de la façon suivante : lorsqu'une même fonction est appelée avec les mêmes
arguments, elle retournera le même résultat.

Haskell étant un langage a usage général, il doit intéragir avec le "monde extérieur"~\footnote{tout ce qui est extérieur au programme}.
Celà a posé un problème puisque le monde extérieur est un monde d'états, qui change~\footnote{lire deux fois le même fichier ou faire deux fois le même appel système ne donnera pas nécessairement le même résultat},
Haskell quand a lui se veut constant.
Il a donc été nécessaire de trouver un moyen de prendre en compte ces changements,
ce qui a donné lieu à beaucoup de travaux sur le \emph{Reactive-based programming}.

Cet investissement massif du monde de la programmation fonctionnelle
dans le domaine à même créer une sous-famille à part entière nommée la
programmation réactive fonctionnelle, \emph{functionnal reactive programming}~\cite{frp}.

De plus cette notion de discrétisation de flux d'un programme a été repris
sous le nom d'event sourcing~\cite{eventsourcing}. 
L'idée étant de prendre une unité logique~\footnote{généralement une classe ou un paquetage},
de fournir une interface permettant la soumission d'événements. Lors de chaque événement
reçu, celui-ci déclenche une série de modifications interne, de manière atomique.

Ce concept a été ensuite remis au goût du jour avec l'introduction du CQRS~\cite{cqrs},
pour \emph{Command Query Responsability Segregation}, lui-même dérivé de
Domain Driven Design~\cite{ddd}.
Il s'agit de stocker directement les événements de chaque unité logique et d'effectuer
des projections~\footnote{une application particulière d'un événement} spécifique à un contexte.

Parmis ses implémentations nous pouvons citer yampa~\cite{yampa}~\cite{arrows};
reactive banana~\cite{reactivebanana} et sodium~\cite{sodium} qui d'une part considèrent que
la variation temporelle est un événement à part entière, contrairement à
la quasi-totalité des autres implantations du fait que la notion de
comportement soit fortement couplé au temps; mais qui plus est les
auteurs se sont aperçus que le coeur de leur code sont le même au nom
près~\cite{sodium_talk}.

\subsection{Actions disponibles}
Les programmes consistent à déclarer des comportements à appliquer à des flux
d'événements.

\subsubsection{\emph{Subscribe / each / map}}
Déclare un comportement sur un flux d'événements.

\subsubsection{\emph{Filter}}
Crée un nouveau flux d'événements respectant un certain nombre de conditions.

Ce nouveau flux comprendra au plus le même nombre d'événements que le flux dont
il est issu.

\subsubsection{\emph{Join}}
Crée un nouveau flux d'événements à partir de plusieurs flux.

Les événements générés seront la concaténation des événements des flux sur lesquels
le comportement est appliqué.

\subsection{Bénéfices}\label{bénéfices-1}

Le premier avantage de l'ordre des événements est la reproductibilité
des situations, ce qui aide considérablement durant la recherche et la
correction de bugs.

De plus, comme le fait le CQRS, il est possible d'intéprêter différemment
les événements, ainsi en cas d'évolution du système, rien n'est perdu et
un grand nombre de données sont déjà présentes.

Enfin, le fait que de nombreux travaux ais été conduits en Haskell sur
le sujet, le système de type à fait que la composabilité, donc
l'extensibilité et la maîtrise de la complexité d'une application, est
au premier plan.

\subsection{Comparaisons aux critères}
\begin{itemize}
    \item[Composabilité] Le soin apporté aux interfaces et au système de types
permet d'avoir des parties génériques et fortement composables.
    \item[Évolutivité] Pour ajouter une nouvelle partie au système, il suffit de
définir une interface et de respecter la signature des interfaces des parties
avec lesquelles la nouvelle partie souhaite communiquer. Ce qui donne une bonne
évolutivité du système.
    \item[Reproductibilité] Il y a reproductibilité du fait de l'apect déterministe
de la réception et du traitement des messages par les différentes parties du système.
\end{itemize}

\section{Comparaison}
Le détail des correspondances de ces solutions avec les critères sera détaillé
dans les sections suivantes, mais voici le récapitulatif :

\begin{center}
\begin{tabular}{r|c c}
Critère & \emph{Flow} & \emph{Reactive} \\
\hline
Composabilité & Mauvaise & Bonne \\
Évolutivité & Mauvaise & Bonne \\
Reproductibilité & Non & Oui \\
\end{tabular}
\end{center}

\section{Lien entre \emph{flow-based programming} et \emph{reactive programming}}
Le \emph{reactive programming} est un sur-ensemble du \emph{flow-based programming}
car il possède des fonctionnalités de bases qui ne sont pas présente en \emph{flow-based programming}.

Ces fonctionnalités sont implantable en \emph{flow-based programming}, de tel sorte
que nous pouvons partir d'un système en \emph{flow-based programming} et le transformer
en système réactif.

\section{Conclusions}\label{conclusions}

Le \emph{data-flow} tente de résoudre des problèmes de vie des systèmes,
notamment de passage à l'échelle en mettant le flux de traitement des données
au centre de la l'architecture logicielle.

Cette étude pourra donc se concentrer sur la nature profonde de la
programmation \emph{data-flow}, en comparant les différentes approches
existante.

D'autres questions comme le typage ou l'expressivité pourront également
se poser durant cette étude.


\section{Références}\label{références}
\bibliographystyle{plain}
\bibliography{refs}

\end{document}
